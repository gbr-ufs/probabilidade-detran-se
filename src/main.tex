% Classe.
\documentclass[
%% Opções da classe memoir.
article, % Indica que é um artigo acadêmico.
12pt,    % Tamanho da fonte.
oneside, % Para impressão apenas no recto.
a4paper, % Tamanho do papel.
%% Opções da classe abntex2.
%%% Tìtulos de letra maiúscula.
chapter=TITLE,
section=TITLE,
subsection=TITLE,
subsubsection=TITLE,
%% Opções do pacote babel.
english,
brazilian
]{abntex2}

% Pacotes.
\usepackage[T1]{fontenc} % Permite fontes com mais glifos (letras).
\usepackage[utf8]{inputenc} % UTF-8.
%% Fonte bonita que é usada até hoje.
%% Veja: <https://tex.stackexchange.com/questions/147194/is-it-still-useful-to-load-the-lmodern-package>
\usepackage{lmodern}
\usepackage{amsmath, amsthm, amssymb, mathtools} % Coisas de matemática.
\usepackage{microtype} % Personalização de justificação.
\usepackage{indentfirst} % Indenta o primeiro parágrafo de cada seção.
\usepackage{graphicx} % Coisas gráficas.
% Comandos
\newcommand{\imprimircapaelogo}{
  \begin{center}
    \includegraphics[scale=0.2]{img/logo_ufs}
  \end{center}
  \imprimircapa
}
\newcommand{\orcid}[2]{
  \href{#1}{\includegraphics[scale=0.06]{img/orcid.pdf}\hspace{1mm}#2}
}
\newcommand{\probabilidadeemt}[1]{
  \hat{P}( \geq 1 \text{ em } t) \text{ com } t = #1
}
% Configuração ABNT.
\titulo{\uppercase{Probabilidade com Dados do DETRAN/SE}}
\autor{\orcid{https://orcid.org/0009-0005-5652-617X}{Gabriel Santos de Souza}}
\orientador{Esdras Adriano Barbosa dos Santos}
\instituicao{Universidade Federal de Sergipe -- UFS}
\local{São Cristóvão}
\data{2025}
\preambulo{Relatório à Universidade Federal de Sergipe como requisito de avaliação parcial da disciplina de Estatística Aplicada e, por conseguinte, a obtenção dos seus créditos.}
% Configuração de citação.
\usepackage[brazilian,hyperpageref]{backref} % Página com citações.
\usepackage[alf]{abntex2cite} % Citações ABNT.
\renewcommand{\backrefpagesname}{Citado na(s) página(s):~}
\renewcommand{\backref}{}
\renewcommand*{\backrefalt}[4]{
  \ifcase #1 %
    Nenhuma citação no texto.%
  \or
    Citado na página #2.%
  \else
    Citado #1 vezes nas páginas #2.%
  \fi}%
% Configuração do PDF.
\makeatletter
\hypersetup{
  pdftitle={\@title},
  pdfauthor={\@author},
  pdfsubject={\imprimirpreambulo},
  pdfcreator={LaTeX with abnTeX2},
  pdfkeywords={universidade federal de sergipe}{federal university of sergipe}{detran}{acidente}{sinistro}{accident}{sergipe},
  colorlinks=true,
  linkcolor=blue,
  citecolor=blue,
  filecolor=magenta,
  urlcolor=blue,
  bookmarksdepth=4
}
\makeatother
% Espaçamento.
\setlrmarginsandblock{3cm}{3cm}{*}
\setulmarginsandblock{3cm}{3cm}{*}
\checkandfixthelayout
\setlength{\parindent}{1.3cm}
\setlength{\parskip}{0.2cm}
\SingleSpacing
% Documento.
\makeindex % Compila o índice.
\begin{document}

\selectlanguage{brazilian}
\frenchspacing % Espaçamento normal entre frases. O LaTeX usa o espaçamento arcaico de 2 espaços em vez de 1.

\imprimircapaelogo

\imprimirfolhaderosto

% Lista de ilustrações.
\pdfbookmark[0]{\listfigurename}{lof}
\listoffigures*

\cleardoublepage

% Lista de tabelas.
\pdfbookmark[0]{\listtablename}{lot}
\listoftables*

\cleardoublepage

\begin{siglas}
\item[DETRAN] Departamento Estadual de Trânsito
\item[SE] Sergipe
\item[SAST] Segurança e Saúde do Trabalho
\item[BPTRAN] Batalhão de Policiamento de Trânsito
\end{siglas}

\tableofcontents* % O "*" imprime o sumário sem indicar a página do próprio sumário.

\cleardoublepage

\textual

\section{Introdução}

Este relatório busca avaliar a situação do transito no estado de Sergipe, particularmente a circulação de motocicletas.

O Conselho Nacional de Trânsito define como ``motocicleta'' um veículo de duas rodas conduzido na posição \textit{montada} (levemente inclinado). Essa categoria difere da motoneta pois estes veículos são conduzidos numa posição diferente, a posição \textit{sentada} \citeonline{venturaVoceSabeQual2025}.

Os dados tem como procedência o SAST/BPTRAN. Outras fontes são disponibilizadas no portal, mas estas são as selecionadas por padrão.

\section{Dados}

80\% dos sinistros de trânsito no Estado tiveram como público motociclistas em 2024 \citeonline{DetranSERealiza2024}.

A frota de motocicletas em Sergipe era de 340889 \citeonline{ministerioFrotaDeVeiculos2024}.

No portal do DETRAN do estado, no ano de 2024, foram registrados 873 sinistros \citeonline{DetranSEEstatisticasSinistro2024}.

\subsection{Cálculos}

As equações usadas aqui podem ser encontradas no documento desta atividade avaliativa, disponibilizado pelo orientador.

\subsubsection{Taxa por Unidade de Exposição (\(\hat{\lambda}\))}

Seja \(Y_{s,v}\), tal que \(s\) é o tipo de sinistro e \(v\) o tipo de veículos, a contagem de eventos em um período.

Seja \(E\) a exposição, que, no caso deste relatório, é a quantidade de veículos \(v\).

Podemos modelar a taxa por unidade de exposição \(\hat{\lambda}\) como

\begin{equation} \label{eq:1}
  \hat{\lambda} = \frac{Y_{s.v}}{E}
\end{equation}

Usando a equação \ref{eq:1} podemos calcular, em ``média'', quantos sinistros envolvendo motocicletas temos por ano:

Usando a informação do portal de 873 sinistros, e as 340889 motocicletas da tabela de frota de veículos do Ministério dos Transportes, temos que

\[
  \hat{\lambda} = \frac{873}{340889} \approx 0,002561 \text{ motocicletas-ano}
\]

\subsubsection{Probabilidades por Ano}

Com essa taxa \(\hat{\lambda}\) em mãos, podemos calcular a probabilidade de uma motocicleta estar envolvida em um sinistro por ano. Através da seguinte equação:

\begin{equation} \label{eq:2}
  \hat{P}(\geq 1 \text{ em } t) = 1 - \exp(-\hat{\lambda}t)
\end{equation}

Onde \(t\) é a quantidade de anos e \(\exp\) é \(\exp(x) = e^{x}\).

Com essa equação \ref{eq:2} em mãos, podemos calcular a probabilidade para diversas quantidades de anos:

\begin{table}[h]
  \ABNTEXfontereduzida
  \caption[Probabilidades de sinistros envolvendo motocicletas (\(t\) = anos)]{Probabilidades de sinistros envolvendo motocicletas (\(t\) = anos).}
  \label{tbl:1}
  \begin{center}
    \begin{tabular}{p{4cm}p{4cm}p{4cm}}
      \toprule
      \(t\) & Probabilidade (\%) & Quantidade de Sinistros \\
      \midrule
      1 & 0,2558\% & 872 \\
      2 & 0,51089\% & 1742 \\
      3 & 0,7654\% & 2609\\
      4 & 1,019\% & 3473\\
      5 & 1,272\% & 4336\\
      \bottomrule
    \end{tabular}
  \end{center}
\end{table}

Demonstrando uma tendência alarmente. Se o governo seguir com campanhas de conscientização tais como a blitz educativa de 29 de novembro \citeonline{DetranSERealiza2024}, essa situação poderá ser revertida.

\section{Quadro}

\begin{table}[h]
  \ABNTEXfontereduzida
  \caption[Quadro de Informações]{Quadro de Informações.}
  \begin{center}
    \begin{tabular}{p{4cm}p{4cm}}
      \toprule
      \textit{Informação} & \textit{Valor} \\
      \midrule
      % \ABNTEXfontereduzida
      Tipo de sinistro (\(s\)) & Sem vítimas \\
      Tipo de veículo (\(v\)) & Motocicleta \\
      Período & Ano 2024 \\
      Município/Âmbito & Sergipe \\
      Contagem \(Y_{s,v}\) & 873 \\
      Exposição \(E\) & 340889 \\
      \(\hat{\lambda} = Y_{s, v}/E\) & 0,00256 \\
      \(\probabilidadeemt{1}\) & 0,2558\% \\
      \(\probabilidadeemt{2}\) & 0,51089\% \\
      \(\probabilidadeemt{3}\) & 0,7654\% \\
      \(\probabilidadeemt{4}\) & 1,019\% \\
      \(\probabilidadeemt{5}\) & 1,272\% \\
      \bottomrule
    \end{tabular}
  \end{center}
  \legend{Fonte: Portal do DETRAN/SE \(\rightarrow\) \textit{Estatísticas de sinistro de trânsito}}
\end{table}

\postextual

\cleardoublepage

\bibliography{bibliografia}

\cleardoublepage

\begin{apendicesenv}
  \partapendices

  \section{Cálculos}

  \subsection{Probabilidades de sinistro envolvendo motocicletas}

  Nesta seção encontram-se os cálculos feitos para cada quantidade de anos da tabela \ref{tbl:1}. A meta foi quatro algarismos significativos, assim como no documento de instrução desta atividade.

  Seja \(P\) a probabilidade, \(Q\) a quantidade e usando a equação \ref{eq:2} para encontrar o valor de \(P\):

  \begin{itemize}
  \item \(t = 1\)
    \[
      P = 1 - \exp(-0,002561) \approx 0,002558 = 0,2558\%
    \]
    \[
      Q = 340889 \cdot 0,2558\% \approx 872
    \]
  \item \(t = 2\)
    \[
      P = 1 - \exp(-2 \cdot 0,002561) \approx 0,0051089 = 0,51089\%
    \]
    \[
      Q = 340889 \cdot 0,51089\% \approx 1742
    \]
  \item \(t = 3\)
    \[
      P = 1 - \exp(-3 \cdot 0,002561) \approx 0,007654 = 0,7654\%
    \]
    \[
      Q = 340889 \cdot 0,7654\% \approx 2609
    \]
  \item \(t = 4\)
    \[
      P = 1 - \exp(-4 \cdot 0,002561) \approx 0,01019 = 1,019\%
    \]
    \[
      Q = 340889 \cdot 1,019\% \approx 3473
    \]
  \item \(t = 5\)
    \[
      P = 1 - \exp(-5 \cdot 0,002561) \approx 0,01272 = 1,272\%
    \]
    \[
      Q = 340889 \cdot 1,272\% \approx 4336
    \]
  \end{itemize}

\end{apendicesenv}

\cleardoublepage

\begin{anexosenv}
  \partanexos

  \begin{figure}[htb]
    \caption[Tabela Completa de Sinistros]{Tabela Completa de Sinistros.}
    \includegraphics[scale=0.6]{img/tabela_completa}
  \end{figure}

  \cleardoublepage

  \begin{figure}[htb]
      \caption[Regras para ciclomotores, motonetas e motocicletas]{Regras para ciclomotores, motonetas e motocicletas.}
      \includegraphics[scale=0.4]{img/ministerio_dos_transportes}
  \end{figure}

\end{anexosenv}

\end{document}
